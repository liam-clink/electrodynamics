\documentclass[12pt,violet]{bbe}
\usepackage{blindtext}

\usepackage{fontspec}
\usepackage{lipsum}

%\setmainfont[Ligatures=TeX]{Cambria}
%\usefonttheme{serif}
%\usefonttheme{professionalfonts}



\begin{document}

\chapter{Electrostatics}
	\section{Coulomb Force}
	What is charge?
	
	\begin{definition}
	$$ \vec{F}_{1 \rightarrow 2} = k \frac{q_1 q_2}{|\vec{r}_2 - \vec{r}_1|^3} (\vec{r}_2 - \vec{r}_1)$$
	\end{definition}
	
	Example: Work out the coulomb force in Cartesian coordinates
	
	\begin{remark}
	This is remark
	\end{remark}
	
	\section{Electric Field}
	The electric field is the same for all charges, while the electric force is charge dependent.
	
	\section{Dipoles}
	
	\begin{definition}
	$$ \vec{p} = \sum_i q_i \vec{r}_i$$
	\end{definition}
	
	The dipole moment is not unique if there is a net charge.
	
	Dipole Field
	
	Force on dipole in electric field
	
	Have a later Chapter about multipole expansion, both for static fields and EM wave modes
	
	
	\section{Polarization Field}
	\begin{remark}
	The distinction between free and bound charges is arbitrary, although examples usually dictate rules of thumb
	\end{remark}
	
	The polarization field is the dipole per volume, and is generally only used when the interaction with materials is being considered.
	
	Example: Calculate dipole of metal ball in uniform electric field, use to calculate susceptibility of metal ball lattice
	
	\blinddocument
	
\end{document}