% !TEX program = lualalatex

\documentclass[12pt]{book}
\usepackage{blindtext}

\usepackage{fontspec}
\usepackage{lipsum}
\usepackage{xcolor}

\usepackage{amssymb,amsmath,amsthm}
\newtheorem{theorem}{Theorem}

\usepackage{amsthm}
\usepackage{physics}



\setmainfont[Ligatures=TeX]{[Cambria]}


\begin{document}

\part{Statics}

\chapter{Electrostatics}
\section{Coulomb Force}
What is charge?

\begin{definition}
	$$ \vec{F}_{1 \rightarrow 2} = k \frac{q_1 q_2}{|\vec{r}_2 - \vec{r}_1|^3} (\vec{r}_2 - \vec{r}_1)$$
\end{definition}

Example: Work out the coulomb force in Cartesian coordinates

\begin{remark}
	This is remark
\end{remark}

\section{Electric Field}
Explain free vs bound right at the start. The electric field is the same for all charges, while the electric force is charge dependent. The $\Vec{E}$ field is the total field, and is what appears in the Coulomb force.
We will start with $\Vec{D}$ field, as we only consider free charges for now.

\section{Dipoles}

\begin{definition}
	$$ \vec{p} = \sum_i q_i \vec{r}_i$$
\end{definition}

The dipole moment is not unique if there is a net charge.

Dipole Field

Force on dipole in electric field

Have a later Chapter about multipole expansion, both for static fields and EM wave modes


\section{Polarization Field}
\begin{remark}
	The distinction between free and bound charges is arbitrary, although examples usually dictate rules of thumb
\end{remark}

The polarization field is the dipole per volume, and is generally only used when the interaction with materials is being considered.

Example: Calculate dipole of metal ball in uniform electric field, use to calculate susceptibility of metal ball lattice


\chapter{Magnetic Fields}
Talk about how demagnetizing field of bar magnet is equivalent to electric field of capacitor.

What about magnetic field doing no work? What about magnetic work on a wire with current flowing?

Talk about how lorentz contraction can't explain wire magnetic force.

\section{Magnetic Induction}

\section{Faraday Paradox}

We've already covered induced EMF in the context of linear motion, but as is usually the case in physics, things can get bizarre when translated to rotational motion.
Unlike when a magnet is linearly translated in the vicinity of a circuit, when a {\color{red} at least disk, don't know about square} magnet is \textit{rotated}, no induced EMF is observed.
This means that the intuition of field lines being cut by a wire is flawed. Pradoxes reveal that our understanding about a situation is incomplete.
Now consider a rotating metal disk, with a brush at the edge, and another at the center, with the brushes connected by a wire that makes this arrangement a closed circuit.
When a magnet is held over the rotating disk, it produces an EMF, but this is something we would expect with the cutting field lines intuition. So what's going on here?
With linear motion, when the magnet and circuit moved together, there was no EMF.
But surprisingly, in the case of rotational motion, when the magnet and this disk are rotating together, there \textit{is} an induced EMF!
This is known as the \textbf{Faraday Paradox}.
This result means that it is not the rotation of the magnet that gives an EMF, but the rotation of the disk. But why?

Well, let's go back to the fundamental equations, rather than relying on intuition.
When a magnet translates in space, the magnetic stray field changes in time, and this change results in a circulation of the electric field according to

$\curl{\vec{E}} = - \pdv{\vec{B}}{t}$.

However, when the magnet is static, and the circuit moves, there is no circulation in the electric field due to the magnet, because the magnetic field is static.
The current that's generated here is produced by the Lorentz force.
The only difference between these scenarios is the change of reference frame of the observer, and the fields do change with reference frame.

	{\color{red} Derive the equivalence between loop integral with lorentz force and change in flux, depending on reference frame.}

In the rotating disk scenario, there is no longer a reference frame where no part of the circuit is moving.

\part{Electrodynamics}

\chapter{A Magnet that Attracts Aluminum and Copper}

\footnote{Discussed in a lecture by Dan Gelbart: https://www.youtube.com/watch?v=7ZeBWJLRXqM}

An alternating magnetic field usually repels non-ferromagnetic conductors.

Copper ring inlaid into steel core of electromagnet.
Since the copper loop experiences a strong alternating magnetic field, it makes an alternating magnetic field that is 90 degrees phase relative to the coil.
The combined field from the coil and the copper ring produces an inwardly rotating toroidal magnetic field.
Geometry of the conductor matters, because if it's large enough to enclose all the field, there's cancellation.

\chapter{Electromagnetic waves}
Talk about mica waveguide cover in microwaves as an example. Mica powder with a phenolic resin binder, high real part of dielectric constant, small imaginary part.
Provides physical protection, allows coupling in of microwaves, but has high enough reflectivity to make the main cavity have nice standing waves without crazy hotspots and such


In wave section, discuss Gaussian beam, and give as a problem finding the intensity isosurfaces, which are dumbbell shaped

\blinddocument

\end{document}