\documentclass[12pt,a4paper,violet]{bbe}
\usepackage{blindtext}

\usepackage{fontspec}
\usepackage{lipsum}

\setmainfont[Ligatures=TeX]{[Cambria.ttf]}


\begin{document}

\chapter{Electrostatics}
	\section{Coulomb Force}
	What is charge? 
	
	\begin{definition}
	$$ \vec{F}_{1 \rightarrow 2} = k \frac{q_1 q_2}{|\vec{r}_2 - \vec{r}_1|^3} (\vec{r}_2 - \vec{r}_1)$$
	\end{definition}
	
	Example: Work out the coulomb force in Cartesian coordinates
	
	\begin{remark}
	This is remark
	\end{remark}
	
	\section{Electric Field}
	Explain free vs bound right at the start. The electric field is the same for all charges, while the electric force is charge dependent. The $\Vec{E}$ field is the total field, and is what appears in the Coulomb force.
    We will start with $\Vec{D}$ field, as we only consider free charges for now.
	
	\section{Dipoles}
	
	\begin{definition}
	$$ \vec{p} = \sum_i q_i \vec{r}_i$$
	\end{definition}
	
	The dipole moment is not unique if there is a net charge.
	
	Dipole Field
	
	Force on dipole in electric field
	
	Have a later Chapter about multipole expansion, both for static fields and EM wave modes
	
	
	\section{Polarization Field}
	\begin{remark}
	The distinction between free and bound charges is arbitrary, although examples usually dictate rules of thumb
	\end{remark}
	
	The polarization field is the dipole per volume, and is generally only used when the interaction with materials is being considered.
	
	Example: Calculate dipole of metal ball in uniform electric field, use to calculate susceptibility of metal ball lattice
	


	Talk about mica waveguide cover in microwaves as an example. Mica powder with a phenolic resin binder, high real part of dielectric constant, small imaginary part.
	Provides physical protection, allows coupling in of microwaves, but has high enough reflectivity to make the main cavity have nice standing waves without crazy hotspots and such
	

	\blinddocument
	
\end{document}